\documentclass{jsarticle}

\usepackage{amsmath}

\begin{document}

PF-AR のパラメータは以下の通り
\begin{table}[h]
\centering
\caption{PF-AR パラメータ一覧\label{tab:pfarparameter}}
\begin{tabular}{cc} \hline
$\sigma_x$ & $2.6~\mathrm{mm}$ \\ 
$\sigma_y$ & $0.16~\mathrm{mm}$ \\ 
$\sigma_z$ & $20~\mathrm{mm}$ \\ 
$q_\mathrm{b}$ & $60~\mathrm{mA} \times 377000~\mathrm{mm}/c = 7.55 \times 10^{-8}~\mathrm{C}$ \\
$\gamma$ & $6.5~\mathrm{GeV}/511~\mathrm{keV} \sim 12720$ \\ \hline
\end{tabular}
\end{table}

坂中さんの博士論文にあるようにバンチを $z$ 方向に一様,$x,\,y$ 方向にガウシアンで分布していると思うと,
\begin{equation}
E_x = \frac{q_\mathrm{b} \cdot x}{2\pi \epsilon_0 l_\mathrm{b}\sigma_x(\sigma_x + \sigma_y)}.
\end{equation}

ただし,ここで$l_\mathrm{b}$ はバンチの$z$ 方向の大きさのスケールである.
これに各値を代入すると,$E_x\,(x = 5\sigma_x)$ として$\sim 2.45 \times 10^6~[\mathrm{V}] /l_\mathrm{b}~[\mathrm{mm}]$ が得られる.
$\pm \sigma_z$ の範囲にある電荷はバンチ全体の半分程度だと考えられるので,$\l_\mathrm{b} = \sigma_z$ を代入し,0.45 をかけると結果として$50~\mathrm{MV/m}$ という値が得られ,これがワイヤの位置が感じる$x$ 方向の電場の最大値であると考えられる.
(ワイヤとバンチが最も近づいたときが最大とする)

次に,バンチが3 次元ガウシアンに従って分布している場合を考える.
このとき
\begin{equation}
E_x = \frac{\gamma}{4\pi\epsilon_0}\frac{q_\mathrm{b}}{2\pi^{3/2}\sigma_x\sigma_y\sigma_z}\iiint_{-\infty}^\infty\frac{(x - \xi)\exp[-\xi^2/(2\sigma_x^2)-\eta^2/(2\sigma_y^2)-\zeta^2/(2\sigma_z^2)]}{[\gamma^2(z-\zeta)^2 + (x - \xi)^2 + (y - \eta)^2]^{3/2}}d\xi d\eta d\zeta.
\end{equation}
これを各$(5\sigma_x, 0, z)$ で計算(しようと)しているのがcalc\_elec である.

\newpage

二次元ラプラス方程式を極座標表示で解くと,その一般解は
\begin{equation}
U(r, \theta) = (A_0\theta + B_0)(C_0\log r + D_0)  + \sum_{n = 1}^\infty (A_n\cos n\theta + B_n\sin n\theta)(C_n r^n + D_n r^{-n}).
\end{equation}

ワイヤの半径を$a$, 誘電率を$\epsilon = \epsilon_r \epsilon_0$ とする (導体の場合は$\epsilon_r\to\infty$ とすればよい).
電場は$x$ 方向にかかっており,バンチのすすむ方向を$y$ 方向としたときに$\theta$ が反時計回りに増加する座標を以降考える.

一様電場$E_0$ (前ページの$E_x$) による電位は遠方で$\phi_\infty = -E_0r\cos\theta$.
ワイヤ外の電位$\phi_o$ はワイヤ内と$\theta$ によらず連続であることと無限遠で$\phi_\infty$ となることから$\phi_o = -E_0(r + \beta/r)\cos\theta$ (ただし,$\beta$ はある定数).

ワイヤ内での電位$\phi_i$ は一般解のうち,まず$\theta$ によらず$\phi_i(a,\theta) = \phi_o(a,\theta)$ が成り立つ必要があるので
\begin{equation}
\phi_i = A\cos\theta (Cr + \frac{D}{r}).
\end{equation}
さらに$r\to 0$ で電位が有限であることから$\phi_i = -E_0r\alpha\cos\theta$ (ただし,$\alpha$ はある定数).


このとき,$\phi_i(a,\theta) = \phi_o(i,\theta)$, $\epsilon_r\partial_r\phi_i(a, \theta) = \partial_r\phi_o(a, \theta)$ を解くと,
\begin{eqnarray}
\alpha a^2 &=& a^2 + \beta, \\
a^2 - \beta &=& \epsilon_r \alpha a^2.
\end{eqnarray}
したがって,$\alpha = 2/(1+\epsilon_r)$, $\beta = a^2(1-\epsilon_r)/(1+\epsilon_r)$ となる.

\begin{eqnarray}
\phi_i &=& - \frac{2 E_0}{1+\epsilon_r} r \cos\theta \\ 
\phi_o &=& -E_0\left(r + \frac{a^2}{r}\frac{1-\epsilon_r}{1+\epsilon_r}\right) \cos\theta
\end{eqnarray}

ワイヤ表面における$r$ 方向の電場は$\partial\phi/\partial r\,(r = a) = -2E_0\cos\theta/(1+\epsilon_r)$, $\theta$ 方向の電場は$r^{-1}\partial\phi/\partial\theta\,(r = a) = 2E_0\sin\theta/(1+\epsilon_r)$ でその大きさは$a$ にはよらない.また,$\epsilon_r$ が大きいものを選ぶほどその絶対値は小さくなる.

\end{document}